%%%%%%%%%%%%%%%%%
% This is an example CV created using altacv.cls (v1.1, 21 November 2016) written by
% LianTze Lim (liantze@gmail.com), based on the 
% Cv created by BusinessInsider at http://www.businessinsider.my/a-sample-resume-for-marissa-mayer-2016-7/?r=US&IR=T
% 
%% It may be distributed and/or modified under the
%% conditions of the LaTeX Project Public License, either version 1.3
%% of this license or (at your option) any later version.
%% The latest version of this license is in
%%    http://www.latex-project.org/lppl.txt
%% and version 1.3 or later is part of all distributions of LaTeX
%% version 2003/12/01 or later.
%%%%%%%%%%%%%%%%

%% If you want to use \orcid or the
%% academicons icons, add "academicons"
%% to the \documentclass options. 
%% Then compile with XeLaTeX or LuaLaTeX.
% \documentclass[10pt,a4paper,academicons]{altacv}
\documentclass[10pt,a4paper]{altacv}

%% AltaCV uses the fontawesome and academicon fonts
%% and packages. 
%% See texdoc.net/pkg/fontawecome and http://texdoc.net/pkg/academicons for full list of symbols.
%% When using the "academicons" option,
%% Compile with LuaLaTeX for best results. If you
%% want to use XeLaTeX, you may need to install
%% Academicons.ttf in your operating system's font %% folder.


% Change the page layout if you need to
\geometry{left=1cm,right=9cm,marginparwidth=6.8cm,marginparsep=1.2cm,top=1cm,bottom=1cm}

% Change the font if you want to.

% If using pdflatex:
\usepackage[utf8]{inputenc}
\usepackage[T1]{fontenc}
\usepackage[default]{lato}
\usepackage{setspace}
\usepackage{hyperref}
\hypersetup{colorlinks=true, urlcolor=blue}

% If using xelatex or lualatex:
% \setmainfont{Lato}

% Change the colours if you want to
\definecolor{DarkBlue}{HTML}{0b2b5f}
\definecolor{Black}{HTML}{000000}

\colorlet{heading}{DarkBlue}
\colorlet{accent}{DarkBlue}
\colorlet{emphasis}{Black}
\colorlet{body}{Black}

% Change the bullets for itemize and rating marker
% for \cvskill if you want to
\renewcommand{\itemmarker}{{\small\textbullet}}
\renewcommand{\ratingmarker}{\faCircle}


%% sample.bib contains your publications
\addbibresource{sample.bib}

\begin{document}
\name{Emanuele Seminara}
  \tagline{Computer Engineer}
% Cropped to square from https://en.wikipedia.org/wiki/Marissa_Mayer#/media/File:Marissa_Mayer_May_2014_(cropped).jpg, CC-BY 2.0
% foto circolare
\photo{5cm}{sagoma}




\personalinfo{%
  % Not all of these are required!
  % You can add your own with \printinfo{symbol}{detail}
  \doublespacing
  \email{\href{mailto:seminara.emanuele96@gmail.com}{seminara.emanuele96@gmail.com}}
  \linkedin{\href{https://www.linkedin.com/in/emanuele-seminara/}{Emanuele Seminara}}
  \github{\href{https://github.com/EmanueleSeminara}{EmanueleSeminara}}
  \homepage{\href{https://emanueleseminara.it}{resume site}}
  \phone{+39 320 4854761}
  \present{19/05/1996}
%   \phone{000-00-0000}
%  \mailaddress{20032, Cormano, MI}
  \location{Italy}
%  \car{Class B}
%  \homepage{}
% I'm just making this up though.
%   \orcid{orcid.org/0000-0000-0000-0000} % Obviously making this up too. If you want to use this field (and also other academicons symbols), add "academicons" option to \documentclass{altacv}
}



%% Make the header extend all the way to the right, if you want. Extend the right margin by 8cm (=6.8cm marginparwidth + 1.2cm marginparsep)
\begin{adjustwidth}{}{-8cm}
\makecvheader
\end{adjustwidth}

%% Provide the file name containing the sidebar contents as an optional parameter to \cvsection.
%% You can always just use \marginpar{...} if you do
%% not need to align the top of the contents to any
%% \cvsection title in the "main" bar.
\cvsection[page1sidebar]{About Me}
It's a pleasure to introduce myself: I'm \textbf{Emanuele}, a \textbf{Computer Engineer} passionate about \textbf{technology} and \textbf{computers} since childhood. During the studies, I deepened my knowledge in \textbf{software development} and developed an interest in \textbf{cybersecurity}. I work as a \textbf{Software Developer} at \href{https://www.elca.ch/}{\textbf{Elca Informatique SA}}. I'm aware that a one-page resume can only provide a glimpse, so feel free to explore \href{https://emanueleseminara.it/}{ my website} for a more comprehensive understanding of my background and skills.

\cvsection{EXPERIENCES}

\cvevent{Full Stack Developer}{Elca Informatique SA}{Apr 2024 - Now}{Palermo}
\begin{itemize}
  \item \textbf{Reservation System Development:} Contributed to the development of a reservation system using Angular for the frontend and Spring for the backend. I created new components for the frontend, managed the audit, and generated Excel reports in the backend. Additionally, I developed frontend tests with Jest, participated in the implementation of end-to-end tests with Cypress, and improved the DevOps pipelines, transitioning from Jenkins to GitHub Actions.

  \item \textbf{Swiss Pension Management System:} Worked on a management system for Swiss pensions with a Spring backend and a JavaFX frontend. I primarily focused on bug fixing for the frontend and the creation of tests for the backend using Mockito.
\end{itemize}

\vspace{1em}

\cvevent{Software Developer}{Progesi S.p.A.}{Aug 2023 - Mar 2024}{Rome}
\begin{itemize}
  \item \textbf{Military Software Development Project:} I was actively involved in the development of a desktop application in \textbf{Java} for ground military operations, \textbf{specializing in collaboration with the main client, Leonardo S.p.A.}. My role encompasses the \textbf{design and implementation of tailor-made software solutions} to address the unique challenges of this context.
  \item \textbf{Military Cloud Service Installation:} I played a crucial role in the \textbf{support and configuration} of an advanced \textbf{Cloud service} within a military center, meticulously managing \textbf{web servers, proxies, and databases}. Ensuring an efficient installation, I ensured that the service fully met the \textbf{specific needs of the military environment} in which it was implemented.
  These experiences have significantly \textbf{enriched my professional skills}, showcasing my ability to tackle \textbf{complex technical challenges} and manage projects in \textbf{highly sensitive environments}.
\end{itemize}

\vspace{1em}

\end{document}
